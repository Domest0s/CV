\documentclass[11pt,a4paper,sans]{moderncv} % Font sizes: 10, 11, or 12; paper sizes: a4paper, letterpaper, a5paper, legalpaper, executivepaper or landscape; font families: sans or roman
\usepackage{tabularx}
\moderncvstyle{classic} % CV theme - options include: 'casual' (default), 'classic', 'oldstyle' and 'banking'
\moderncvcolor{blue} % CV color - options include: 'blue' (default), 'orange', 'green', 'red', 'purple', 'grey' and 'black'


\usepackage[scale=0.8]{geometry} % Reduce document margins
%\setlength{\hintscolumnwidth}{3cm} % Uncomment to change the width of the dates column
%\setlength{\makecvtitlenamewidth}{10cm} % For the 'classic' style, uncomment to adjust the width of the space allocated to your name

%----------------------------------------------------------------------------------------
%   NAME AND CONTACT INFORMATION SECTION
%----------------------------------------------------------------------------------------

\firstname{Bereziuk} % Your first name
\familyname{Ivan} % Your last name
	
% All information in this block is optional, comment out any lines you don't need
%\title{Curriculum Vitae}
\address{Prinses Wilhelminastraat 45 }{2313 AV, Leiden}{Netherlands}
\mobile{+31 6 1372 1358}
\extrainfo{
	Birth date: 07 Jul 1992\\
	GitHub: \url{github.com/Domest0s} \\
	Citizenship: Ukraine}

%\phone{(000) 111 1112}
%\fax{(000) 111 1113}
\email{Navi.Bereziuk@gmail.com}

%\homepage{staff.org.edu/~jsmith}{staff.org.edu/$\sim$jsmith} % The first argument is the url for the clickable link, the second argument is the url displayed in the template - this allows special characters to be displayed such as the tilde in this example
%\extrainfo{additional information}
\photo[120pt][0.4pt]{picture} % The first bracket is the picture height, the second is the thickness of the frame around the picture (0pt for no frame)
%\quote{"A witty and playful quotation" - John Smith}

%----------------------------------------------------------------------------------------

\begin{document}

\makecvtitle % Print the CV title


%----------------------------------------------------------------------------------------
%   EDUCATION SECTION
%----------------------------------------------------------------------------------------

\section{Education}

\cventry{2013-2015}{Master of Physics}{Taras Shevchenko National University of Kyiv}{Ukraine}{}{}
%EQF~level~7
\cvitem{}{Thesis title: \emph{Initial design studies of a drift tube detector for SHiP experiment, CERN.}}
\cvitem{}{Supervisor: Dr. Massimiliano Ferro-Luzzi, massimiliano.ferro-luzzi@cern.ch}
\cvitem{}{Studied and simulated the behaviour of long STRAW tubes for SHiP experiment. The project was developed at CERN.} %under influence of gravity and electromagnetic field, signal gain, time-dependent characteristics, tube efficiency and precision that tubes can grant in SHiP experiment using GARFIELD software. Also this thesis contain comparison of tube gain and efficiency for tube samples with with theoretical prediction for tube gain and efficiency from GARFIELD simulation.}

\cventry{2009-2013}{Bachelor of Physics}{Taras Shevchenko National University of Kyiv}{Ukraine}{}{}	
\cvitem{}{Thesis title: \emph{Position sensitive Si microstrip detector for X-ray analysis}}
\cvitem{}{Supervisor: Prof. Dr. Valery Pugatch, pugatch@kinr.kiev.ua}
\cvitem{}{Made the restore work and reverse engineering for Si-strip sensor-based detector equipment (ASIC + FPGA based DAQ) and further developing of the prototype.}{}{}{}{}


%----------------------------------------------------------------------------------------
%   WORK EXPERIENCE SECTION
%----------------------------------------------------------------------------------------

\section{Work \& Practical Experience}

%\subsection{Vocational}

\cventry{May 2016 - Oct 2016}{Software-engineer}{\textsc{CreaLog}}{Munich}{Germany}{
\begin{itemize}
	\item Developed a JavaScript library as a part of a web application where user constructs a flowchart. This is a highly interactive and easy-to-use application with intuitive interface (based on Drag\&Drop gestures).
\end{itemize}
}
%------------------------------------------------

\cventry{Oct 2015 - Dec 2015}{Intern}{\textsc{Samsung R\&D Institute Ukraine}}{Kyiv}{}{
\begin{itemize}
	\item Developed mobile application under TIZEN OS (\textbf{C/C\texttt{++}} programming). The mobile application is for sensitive data storing (information such as passwords, logins etc). To login to the app, it uses facial recognition which can be set up using a ``selfie''. The app is now on the market.
	\item Had an experience of using computer vision library OpenCV.
	\item Used Git, TDD, Agile Software Development process.
\end{itemize}
}
%------------------------------------------------

\cventry{July 2014 - Jan 2015}{Intern}{\textsc{EPFL}}{Lausanne}{Switzerland}{
\begin{itemize}
	\item Developing, documenting and maintenance of a \texttt{C++} library responsible for parsing and analyzing of \LaTeX -article's source code.
	\item Project placed here: \url{github.com/Domest0s/texpp}
\end{itemize} 
}

%------------------------------------------------

\cventry{Nov 2013 - Feb 2014}{iOS mobile developer}{SAPRUN}{Kyiv}{Ukraine}{
\begin{itemize}
	\item Development of business oriented mobile(iOS) application (Objective-C).
\end{itemize} }

\cventry{July 2012 - Sept 2012}{Intern at ``DESY Summer Student Programme''}{Zeuthen}{Germany}{}{
\begin{itemize}
	\item Used ANSYS software for destructive analysis and modeling of photon penetrating target - source of positrons at International Linear Collider.
\end{itemize} 
}
\clearpage


%----------------------------------------------------------------------------------------
%   PERSONAL SKILLS
%----------------------------------------------------------------------------------------

\section{Personal skills}
\renewcommand{\arraystretch}{1.5}
\setlength{\tabcolsep}{5pt}
\cvitem{Languages}{ 
	English (professional working proficiency) \newline{}
	German (basic (A2.1)) \newline{}
	Ukrainian (mother tongue) \newline{}
	Russian (fluent)
}

\subsection{Technology summary}

\cvitem{OS}{Ubuntu, Windows, Mac OS X}
\cvitem{Programming languages}{
%		\texttt{\textbf{C/C++11} (+STL \& boost)} & advanced \\
%		\LaTeX , ActionScript 3.0, ROOT, Mathematica, MATLAB & good \\
%		Objective-C, Verilog, FORTRAN, SQL, QML, Java, Python, Pascal & basic \\
%		\end{tabular} \newline
	advanced: \texttt{\textbf{C/C++11} + (STL \& boost)}, JavaScript \newline 
	good:~~~~~ \LaTeX , ActionScript 3.0, ROOT, Mathematica, MATLAB \newline
	basic:~~~~~ Objective-C, Java, Verilog, FORTRAN, SQL, QML, Python, Pascal
	}
\cvitem{Additional skills}{Git, CMake, bash, HTML, CSS, SVG, XML, ExtJS, Doxygen, LabVIEW, PhotoShop}
\cvitem{Target programming}{TIZEN OS mobile, Arduino, STM32, ATmega8,  FPGA(Altera, Xilinx)}

\cvitem{Digital competence}{
	\begin{tabularx}{0.9\textwidth}{ 
		>{\setlength\hsize{1\hsize}\centering}X
		|>{\setlength\hsize{1\hsize}\centering}X
		|>{\setlength\hsize{1\hsize}\centering}X
		|>{\setlength\hsize{1\hsize}\centering}X
		|>{\setlength\hsize{1\hsize}\centering}X } 
%  	\hline
	Information processing & Communication & Content creation & Safety & Problem solving \tabularnewline
	\hline 
  	Proficient user & Proficient user & Basic user & Independent user & Independent user \tabularnewline
%	\hline
\end{tabularx}
}


%----------------------------------------------------------------------------------------
%   INTERESTS SECTION
%----------------------------------------------------------------------------------------

\section{Interests}
\cvitem{technical}{system programming, OpenCL, multithreading, high performance computing, 3D~API (DirectX/openGL/Vulcan),  Microcontrollers \& FPGA-programming, programmable technique}
\cvitem{Hobbies}{Basketball, Guitar
	}

\section{References}
\cvitem{CreaLog Munich}
{Dr. Bernd Plannerer, Tel: \texttt{+}49 89 324 656-13, Email: Bernd.Plannerer@crealog.com}
\cvitem {Samsung R\&D Institute Ukraine}
	{Oleg Kopysov, Tel: \texttt{+}38 066 6759 683, Email: o.kopysov@samsung.com, 
	\newline {LinkedIn profile: \url{https://www.linkedin.com/in/oleg-kopysov-a2a50371}}
% my sigh
	\newline{} \newline{} \newline{} \newline{}
	\includegraphics[width=0.1\textwidth]{signatureIB.jpg}	
	\newline{} \newline{}	
	Leiden, \hspace{20pt} \today  \newline{}	
}
%----------------------------------------------------------------------------------------

\end{document}